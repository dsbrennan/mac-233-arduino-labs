\documentclass[11pt,a4paper]{article}
\usepackage[top=3cm, bottom=3cm, left=2.5cm, right=2.2cm]{geometry} %geometry of page
\usepackage[hidelinks]{hyperref} %reference links
\usepackage{fancyhdr} %header and footer control
\usepackage{qrcode} %qrcode links
\usepackage{tabularx} %tables
\usepackage{multirow} % multirow cells
\usepackage{tikz} %diagrams
\usepackage[svgnames]{xcolor} % colours
\usetikzlibrary{backgrounds} % tikz layers

\linespread{1.3}
\emergencystretch 3em

% Set Header and Footer
\fancyhead{}
\fancyhead[L]{\textbf{Lab: GTA Guide}}
\fancyhead[R]{d.s.brennan@sheffield.ac.uk}
\fancyfoot{}
\fancyfoot[L]{\thepage}
\fancyfoot[R]{MAC 233 Arduino Labs, School of MAC, University of Sheffield}

% Create Document
\begin{document}
\pagestyle{fancy}

%%%%%%%%%%%%%%%%%%%%%%%
%% Lab 1: Morse Code %%
%%%%%%%%%%%%%%%%%%%%%%%

\section*{Lab 1: Morse Code}

\begin{tabularx}{\textwidth}{l@{\hspace{1em}}|@{\hspace{1em}}l}
    Source Code & Documentation\\
    \qrcode[hyperlink, height=.31\textwidth]{https://github.com/dsbrennan/ mac-233-arduino-labs/blob/main/lab-1-blink/lab-1-blink.ino}
    &
    \qrcode[hyperlink, height=.31\textwidth]{https://github.com/dsbrennan/ mac-233-arduino-labs/blob/main/lab-1-blink/docs/lab-1-blink.pdf} \\
\end{tabularx}

\subsection*{FAQ's}
Coming Soon

%%%%%%%%%%%%%%%%%%%%%%%%%
%% Lab 2: External LED %%
%%%%%%%%%%%%%%%%%%%%%%%%%

\section*{Lab 2: External LED}

\begin{tabularx}{\textwidth}{l@{\hspace{1em}}|@{\hspace{1em}}l}

    Wiring Diagram & Source Code \\

    \multirow{3}{*}[6.5em]{
        \begin{tikzpicture}[
            scale=1.75, transform shape,
            wire/.style={line width=.8mm},
            leg/.style={draw=DimGray},
            jumper/.style={draw=LimeGreen},
        ]

            %% legs 5 per span y = 0.188, per span x = 0.183
            \draw[wire, leg] (.975,-0.515) -- (.975, 0.792);
            \draw[wire, leg] (.792,-1.61) -- (.792, -0.49);

            %% cables
            \draw[wire, jumper] (.607,-1.61) -- (.607, 0.604);

            %% components
            \filldraw[fill=red, draw=black] (.79,-1.05) circle (1.25mm);
            \node[] at (.975, 0.14) {\includegraphics[angle=90, origin=c,width=1.75mm]{./images/resistor-220.png}};
            \node[] at (-.13, 1.43) {\includegraphics[angle=90, origin=c, width=3.38em]{./images/nano.png}};

            %% breadboard background
            \begin{scope}[on background layer]
                \node[] at (0, 0) {\includegraphics[width=10em]{./images/breadboard.png}};
            \end{scope}

        \end{tikzpicture}
    } 

    & \qrcode[hyperlink, height=.31\textwidth]{https://github.com/dsbrennan/mac-233-arduino-labs/blob/main/lab-2-led/lab-2-led.ino}
    \\

    & Documentation \\

    & \qrcode[hyperlink, height=.31\textwidth]{https://github.com/dsbrennan/mac-233-arduino-labs/blob/main/lab-2-led/docs/lab-2-led.pdf}\\
    
\end{tabularx}

\subsection*{FAQ's}
Coming Soon


%%%%%%%%%%%%%%%%%%
%% Lab 3: Servo %%
%%%%%%%%%%%%%%%%%%

\section*{Lab 3: Servo}

\begin{tabularx}{\textwidth}{l@{\hspace{1em}}|@{\hspace{1em}}l}

    Wiring Diagram & Source Code \\

    \multirow{3}{*}[6.5em]{
        \begin{tikzpicture}[
            scale=1.75, transform shape,
            wire/.style={line width=.8mm},
            jumper/.style={draw=LimeGreen},
        ]

            %% mask out breadboard
            \filldraw[fill=white, draw=white] (0.54, -1.9) rectangle (1.04, -1.5);

            %% cables
            \draw[wire, jumper] (0.69, -1.75) -- (.607,-1.5) -- (.607, 0.604);
            \draw[wire, jumper] (0.792, -1.75) -- (.792, -0.49) to[out=130, in=50] (.42, -0.49) -- (-.87, -0.49) -- (-.87, 0.604);
            \draw[wire, jumper] (0.89, -1.75) -- (.975,-1.5) -- (.975, 2.08);

            %% components
            \draw[wire, draw=brown] (0.69,-2.32) -- (0.69, -1.75);
            \draw[wire, draw=red] (0.79,-2.32) -- (0.79, -1.75);
            \draw[wire, draw=orange] (0.89,-2.32) -- (0.89, -1.75);
            \filldraw[fill=black] (0.6, -1.8) rectangle (0.98, -1.6);
            \filldraw[fill=blue, draw=black] (-.15, -2.76) rectangle (1.05, -2.27);
            \filldraw[fill=white] (0.2,-2.51) circle (2mm);
            \node[] at (-.13, 1.43) {\includegraphics[angle=90, origin=c, width=3.38em]{./images/nano.png}};

            %% breadboard background
            \begin{scope}[on background layer]
                \node[] at (0, 0) {\includegraphics[width=10em]{./images/breadboard.png}};
            \end{scope}

        \end{tikzpicture}
    } 

    & \qrcode[hyperlink, height=.31\textwidth]{https://github.com/dsbrennan/mac-233-arduino-labs/blob/main/lab-3-servo/lab-3-servo.ino}
    \\

    & Documentation \\

    & \qrcode[hyperlink, height=.31\textwidth]{https://github.com/dsbrennan/mac-233-arduino-labs/blob/main/lab-3-servo/docs/lab-3-servo.pdf}\\
    
\end{tabularx}

\subsection*{FAQ's}
Coming Soon

%%%%%%%%%%%%%%%%%%%%%%%%
%% Lab 4: Hall Effect %%
%%%%%%%%%%%%%%%%%%%%%%%%

\section*{Lab 4: Hall Effect Sensors}

\begin{tabularx}{\textwidth}{l@{\hspace{1em}}|@{\hspace{1em}}l}

    Wiring Diagram & Source Code \\

    \multirow{3}{*}[6.5em]{
        \begin{tikzpicture}[
            scale=1.75, transform shape,
            wire/.style={line width=.8mm},
            leg/.style={draw=DimGray},
            jumper/.style={draw=LimeGreen},
        ]

            %% mask out breadboard
            \filldraw[fill=white, draw=white] (-.62, -1.49) rectangle (-.44, -0.98);

            %% legs 5 per span y = 0.188, per span x = 0.183
            \draw[wire, leg] (.975,-0.515) -- (.975, 1.16); %220 resistor
            \draw[wire, leg] (.792,-1.61) -- (.792, -0.49);%led
            \draw[wire, leg] (-.495,-1.61) -- (-.495, -0.86);%hall effect
            \draw[wire, leg] (-.69,-1.23) -- (-.59, -1.23);%hall effect
            \draw[wire, leg] (-.865,-1.61) -- (-.865, -0.86);%10k resistor

            %% cables
            \draw[wire, jumper] (.607,-1.61) -- (.607, 0.604);
            \draw[wire, jumper] (-1.05,-1.244) -- (-1.05, 0.238);
            \draw[wire, jumper] (-.31,-0.88) -- (-.31,-0.32) -- (-.865,-0.32) -- (-.865, 0.604);
            \draw[wire, jumper] (-.31,-1.61) -- (.422,-1.61) -- (.422,-0.307) to[out=50, in=130] (.792,-0.307) -- (.792, 0.792);

            %% components
            \node[] at (-.865, -1.23) {\includegraphics[angle=90, origin=c,width=1.75mm]{./images/resistor-10k.png}}; %10k resistor
            \fill[fill=red] (-.44,-1.44) -- (-.44, -1.04) -- (-.54, -1.04) -- (-.59, -1.14) -- (-.59, -1.34) -- (-.54, -1.44) -- cycle; %hall effect
            \filldraw[fill=red, draw=black] (.79,-1.05) circle (1.25mm); %led
            \node[] at (.975, 0.323) {\includegraphics[angle=90, origin=c,width=1.75mm]{./images/resistor-220.png}}; %220 resistor
            \node[] at (-.13, 1.43) {\includegraphics[angle=90, origin=c, width=3.38em]{./images/nano.png}}; %nano

            %% breadboard background
            \begin{scope}[on background layer]
                \node[] at (0, 0) {\includegraphics[width=10em]{./images/breadboard.png}};
            \end{scope}

        \end{tikzpicture}
    } 

    & \qrcode[hyperlink, height=.31\textwidth]{https://github.com/dsbrennan/mac-233-arduino-labs/blob/main/lab-4-hall-effect-sensor/lab-4-hall-effect-sensor.ino}
    \\

    & Documentation \\

    & \qrcode[hyperlink, height=.31\textwidth]{https://github.com/dsbrennan/mac-233-arduino-labs/blob/main/lab-4-hall-effect-sensor/docs/lab-4-hall-effect-sensor.pdf}\\
    
\end{tabularx}

\subsection*{FAQ's}
Coming Soon


%%%%%%%%%%%%%%%%%%%%%%%%%%%
%% Lab 5: Control System %%
%%%%%%%%%%%%%%%%%%%%%%%%%%%

% \section*{Lab 5: Control System}

% \begin{tabularx}{\textwidth}{l@{\hspace{1em}}|@{\hspace{1em}}l}

%     Wiring Diagram & Source Code \\

%     \multirow{3}{*}[6.5em]{

%     } 

%     & \qrcode[hyperlink, height=.31\textwidth]{https://github.com/dsbrennan/mac-233-arduino-labs/blob/main/lab-5-control-system/lab-5-control-system.ino}
%     \\

%     & Documentation \\

%     & \qrcode[hyperlink, height=.31\textwidth]{https://github.com/dsbrennan/mac-233-arduino-labs/blob/main/lab-5-control-system/docs/lab-5-control-system.pdf}\\
    
% \end{tabularx}

% \subsection*{FAQ's}


\section*{Images}
Original images for the wiring is taken from:
\begin{itemize}
    \item Breadboard (\url{https://freesvg.org/breadboard})
    \item Resistor (\url{https://freesvg.org/resistor-330-ohm})
\end{itemize}

\end{document}